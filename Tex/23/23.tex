\pagestyle{fancy}
\renewcommand{\theUnit}{8}
\ifthenelse{\isundefined{\UnitPageNumbers}}{}{\setcounter{page}{1}}
\rhead{Chapter  \theUnit: Hypothesis Tests}
\lhead{Math 3382: Statistical Theory}
%\lhead{\includegraphics[width=1.25cm]{CUDenver-Logo.png}}
\rfoot{\mypage}
\cfoot{\includegraphics[width=2.25cm]{CUDenver-Logo-coverpage.png}}
\lfoot{Adam Spiegler}
\fancypagestyle{firstfooter}{\footskip = 50pt}
\renewcommand{\footrulewidth}{.4pt}
%%%%%%%%%%%%%%%%%%%%%%%%%%%
\vspace*{-20pt} \thispagestyle{firstfooter}


%\begin{tasks}[counter-format = {(tsk[a])},label-offset = {0.8em},label-format = {\color{black}\bfseries}](2)
\pagebegin{Section 8.4: Type I and Type II Errors}

\bb
\ii A company claims that only 3\% of people who use their facial lotion develop an allergic reaction (a rash). You are suspicious of their claim based on hearing stories from students in your classes, and you believe it is more than 3\%. You pick a random sample of 50 people and have them try the lotion. If more than 3 out of the 50 people develop the rash, you will blow up social media with posts about the dishonesty of the company's claim.

\bb
\ii Set up hypotheses for this test. \vspace{0.6in}
\ii Explain what  Type I and Type II errors are in this case. Make sure you explain in the context of this example. \vspace{1in}
\ii What is the probability of making a Type I error? What might be the ramifications? \vfill
\ii If you were to perform the hypothesis test at a 5\% significance level, and you observe $X=4$, what would be the result of the test? \vfill
\ii For what values of $X$ would you reject $H_0$ at a 5\% significance level? \vfill
\ee
\ee

\clearpage

\pagebegin{Rejection Region}


\bbox
When performing a hypothesis test at a significance level of $\alpha$, the \textbf{\colorb{rejection or critical region}}, $\mathcal{R}$ is the set of all values of the test statistic for which we reject $H_0$. The endpoint(s) of the region are called \textbf{\colorb{critical values}}.
\ebox

\bb[resume]
\ii We tested to whether if the ten-sided die is fair or not by rolling it 20 times, and counting the number of rolls that land on an even number. If $p$ is the proportion of all rolls that land on an even number, then we had
\[ H_0: p = 0.5 \ \ \ \ \mbox{vs.} \ \ \ \ \ H_a: p \ne 0.5. \]
\bb
%\ii If you found only $X=7$ rolls landed on an even number and believe that is enough to reject $H_0$. What is the probability of making a Type I error? \vspace{3in}
\ii If you found only $X=7$ rolls landed on an even number, what is the P-value? \vspace{3in}
\ii Find the critical values and rejection region If we use a significance level of 10\%.
\ee

\clearpage

\pagebegin{Section 8.4.2: Power of a Test}

\ii Suppose you are interested in the lengths of a certain species of snake present around Lake Randazzo. Assume the lengths (in cm.) are normally distributed with unknown mean $\mu$, but the standard deviation of the population is known to be $\sigma = 4$ cm. It has been claimed that the mean length of this species is 25 cm. You believe the actual mean length is greater than 25 cm. You collect a random sample of 30 snakes. You will test using a significance level of $\alpha = 0.05$.
\bb
\ii Set up hypotheses for the test. \vspace{0.6in}
\ii Find the critical value, and give the rejection region. \vfill
\ii If in fact $\mu = 27$ cm, what is the probability of making a Type II error? \vspace{1.5in}
\ii What is the probability of correctly rejecting $H_0$ when $H_a$ is true? \vspace{1in}
\ee
\ee

\bbox
The \textbf{\colorb{power}} of a test is the probability of correctly rejecting $H_0$.
\[ \mbox{power} = P(\mbox{Reject $H_0$ \ $|$ \ $H_a$ is true}) = 1 - \beta ,\]
where $\beta$ denotes the probability of a Type II error.
\ebox

\clearpage

\bb[resume]

\ii Let $X_1$, $X_2$, \ldots $X_{12}$ be a random variable from a Bernoulli distribution with unknown probability $p$. We test $H_0: p=0.3$ versus $H_a: p < 0.3$. We will reject the null if the number of success $Y= X_1 + X_2 + \ldots + X_{12} \leq 1$.
\bb
\ii Find the probability of a Type I error.
\ii If the alternative hypothesis is true, find an expression for the power, $1-\beta$, as a function of $p$.
\ee

\vfill

\ii You draw a random sample $X_1, X_2, \ldots , X_{10}$ from an exponential distribution $f(x; \lambda) = \lambda e^{-\lambda x}$ (recall $\mu = 1/\lambda$). You will test $H_0: \lambda = 0.25$ versus $H_a: \lambda < 0.25$. You decide you will reject the null hypothesis if at least 3 of the values  of $X_i$ are greater than 9.
\bb
\ii Compute the probability of a Type I error.
\ii If actually $\lambda = 0.15$, what is the power of this test?
\ee
\vfill
\ee
